%%%%%%%%%%%%%%%%%%%%%%%%%%%%%%%%%%%%%%%%%
% a0poster Portrait Poster
% LaTeX Template
% Version 1.0 (22/06/13)
%
% The a0poster class was created by:
% Gerlinde Kettl and Matthias Weiser (tex@kettl.de)
% 
% This template has been downloaded from:
% http://www.LaTeXTemplates.com
%
% License:
% CC BY-NC-SA 3.0 (http://creativecommons.org/licenses/by-nc-sa/3.0/)
%
%%%%%%%%%%%%%%%%%%%%%%%%%%%%%%%%%%%%%%%%%

%----------------------------------------------------------------------------------------
%	PACKAGES AND OTHER DOCUMENT CONFIGURATIONS
%----------------------------------------------------------------------------------------

\documentclass[a0,portrait]{a0poster}

\usepackage{multicol} % This is so we can have multiple columns of text side-by-side
\columnsep=100pt % This is the amount of white space between the columns in the poster
\columnseprule=3pt % This is the thickness of the black line between the columns in the poster

\usepackage[svgnames]{xcolor} % Specify colors by their 'svgnames', for a full list of all colors available see here: http://www.latextemplates.com/svgnames-colors

\renewcommand{\familydefault}{\sfdefault}

\usepackage{graphicx}
\usepackage{amssymb, amsmath}
\usepackage{xfrac}
\usepackage[section]{placeins}
\numberwithin{equation}{section}
%\numberwithin{figure}{section}
\usepackage{wrapfig}
\usepackage{caption}
%\usepackage{subcaption}
\usepackage[caption=false]{subfig}
\usepackage[font=small,labelfont=bf]{caption} % Required for specifying captions to tables and figures
\graphicspath{{figures/}}
\usepackage{xcolor, colortbl}
\usepackage{adjustbox}
\usepackage{array,multirow}
\usepackage{float}
\usepackage{libertine}
\usepackage{libertinust1math}
%\usepackage[T1]{fontenc}
\usepackage{algpseudocode}
\usepackage{algorithm}
\usepackage{booktabs}
\usepackage{lipsum}
\newcolumntype{R}[2]{%
	>{\adjustbox{angle=#1,lap=\width-(#2)}\bgroup}%
	l%
	<{\egroup}%
}
\newcommand*\rot{\multicolumn{1}{R{90}{1em}}}% no optional argument here, please!


\definecolor{redish}{rgb}{0.8,0.5,0}
\definecolor{gray2}{gray}{0.5}
\definecolor{gray}{rgb}{.8,.8,.8}


%\newcommand{\gray}{\cellcolor{gray}}
\newcommand{\redt}{\textcolor{redish}}
\newcommand{\redm}{\textcolor{red}}
\def\comment{\textcolor{red}}{{}}

\begin{document}

%----------------------------------------------------------------------------------------
%	POSTER HEADER 
%----------------------------------------------------------------------------------------

% The header is divided into two boxes:
% The first is 75% wide and houses the title, subtitle, names, university/organization and contact information
% The second is 25% wide and houses a logo for your university/organization or a photo of you
% The widths of these boxes can be easily edited to accommodate your content as you see fit

\begin{minipage}[b]{0.8\linewidth} % make this 0.7 if there is also a QR code
\veryHuge \color{NavyBlue} \textbf{Title \\ more if title is long} \color{Black}\\[2cm] % Title 
%\Huge\textit{An Exploration of Complexity}\\[2cm] % Subtitle
\huge \textbf{Authors}\\[0.5cm] % Author(s)
\small ARTORG Center for Biomedical Engineering Research, Artificial Intelligence in Medical Imaging Laboratory, Universit\"{a}t Bern, Switzerland % [0.4cm] University/organization
%  \Large \texttt{john@LaTeXTemplates.com} --- 1 (000) 111 1111\\
\end{minipage}
\begin{minipage}[b]{0.2\linewidth}
	\includegraphics[width=12cm]{logos/unibe}
\end{minipage}
% uncomment this if you have a QR code
%\begin{minipage}[b]{0.1\linewidth}
%	\includegraphics[width=6cm]{qrcode}
%\end{minipage}


\vspace{1cm} % A bit of extra whitespace between the header and poster content

%----------------------------------------------------------------------------------------

\begin{multicols}{2} % This is how many columns your poster will be broken into, a portrait poster is generally split into 2 columns

\lipsum[1]
\begin{figure}[H]
	\centering
	\includegraphics[width=0.2\textwidth]{placeholder}
	\caption{A Figure}
\end{figure}
\section*{Method}
\begin{figure}[H]
	\centering
	\includegraphics[width=0.2\textwidth]{placeholder}
	\caption{Yet another figure}
\end{figure}
\lipsum[1]

\subsection*{Some title}
\lipsum[1]


\begin{minipage}[c]{0.2\linewidth}
A list:
\begin{itemize}
	\item item 1
	\item item 2
	\item item 3
	\item item 4
\end{itemize}
\end{minipage}
\begin{minipage}[c]{0.8\linewidth}
	\begin{figure}[H]
		\centering
		\includegraphics[width=0.5\textwidth]{placeholder}
		\caption{Figure again, but in minipage.}
	\end{figure}
\end{minipage}
\\[0.6cm]
An equation for triangles
$$
 a^2+b^2=c^2 
$$
\section*{Experiments and results}
\lipsum[2]
\begin{center}
	\begin{tabular}{lcc}
		\toprule
		& \textbf{bla} & \textbf{bla}\\
		\midrule
		\textbf{bli} & 1 & 2 \\
		\textbf{blo} & 3 & 4 \\
		\bottomrule
	\end{tabular}
\end{center}
\input{Conclusions.tex}

%\nocite{*} % Print all references regardless of whether they were cited in the poster or not
\small{
\bibliographystyle{ieeetr}
\bibliography{refs}
\textbf{Funding:} This work was partly funded by the Swiss National Science Foundation grant No. xxx.}
\end{multicols}
\end{document}